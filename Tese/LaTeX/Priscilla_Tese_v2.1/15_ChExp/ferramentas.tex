\section{Ferramentas}

Com o crescimento da pesquisa sobre aprendizado em FD, é interessante o investimento em ferramenta de software para a aplicação das diversas técnicas propostas. Existem hoje ferramentas, disponíveis gratuitamente, como:

\begin{itemize}
    \item \textbf{MOA (\emph{Massive On-line Analysis}} \cite{MOA} - um \emph{framework} de código aberto que disponibiliza implementação de uma série de algoritmos e métricas para classificação, principalmente, e para agrupamento em FD. A ferramenta também conta com recursos para visualização dos processos de aprendizado.
    
    \item \textbf{VFML (\emph{Very Fast Machine Learning})} \cite{VFML} - um pacote de implementações para mineração de FD de alta velocidade e conjuntos de exemplos \emph{very large}.% Está disponível pelo endereço: http://www.cs.washington.edu/dm/vfml/.
\end{itemize}

A linguagem R \cite{linguagemR} vem sendo muito utilizada para análise de dados e aprendizado de máquina, ... Foi escolhida para implementação da proposta e execução de experimentos.

O pacote \textbf{stream} \cite{lingR_Stream} ... R

O pacote \textbf{streamMOA} \cite{lingR_StreamMOA} ...  interface java moa  -- R

O pacote \textbf{e1071} \cite{lingR_e1071} ... FCM
