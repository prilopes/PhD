\section{Aprendizado Semissupervisionado}  \label{ChAM:semissupervisionado}

A ideia de exploração de informações rotuladas e não rotuladas pelo mesmo processo de aprendizado, chamado aprendizado semissupervisionado, não é atual \cite{Pedrycz1985, Board1989}, mas vem sendo mais explorada, principalmente, na última década \cite{Chapelle2006, Schwenker2014}.
 
O aprendizado semissupervisionado tem como base técnicas supervisionadas ou não supervisionadas, adaptadas a fim de realizar a aprendizagem utilizando conjuntos parcialmente rotulados e/ou algum outro tipo de informação prévia já disponível.

Um número crescente de publicações e conferências sobre aprendizado semissupervisionado pode ser observado, sendo que as técnicas propostas têm sido aplicadas com sucesso, especialmente, em processamento de imagens \cite{Bensaid1996,Grira2006,Pedrycz2008} e classificação de textos \cite{Liu2003,Geng2009}.

As publicações sugerem e analisam modificações de métodos já conhecidos a fim de considerar sua aplicação a um conjunto com maioria de dados não rotulados e uma pequena parte de dados rotulados. A obra de \citeonline{Zhu2009} apresenta de forma resumida algumas tendências e características para classificação semissupervisionada, como \emph{self-training}, \emph{co-training} e \emph{generative models} \cite{Chapelle2006}, e apontamentos a respeito de outras formas de aprendizado semisupervisionado, como por agrupamento.

A utilização de métodos de agrupamento em aprendizado semissupervisionado pode ocorrer de duas formas: colaboração na rotulação do conjunto de exemplos ou agrupamento considerando informação prévia. No primeiro caso, algoritmos de agrupamento são aplicados ao conjunto de exemplos não rotulado para gerar grupos que, posteriormente, serão rotulados por algum outro método, com base na porção rotulada do conjunto. No segundo caso, métodos consagrados de agrupamento são modificados a fim de implementar a semissupervisão já no processo de geração de grupos e, em alguns casos, poder definir rótulos para estes grupos.

Chama-se de agrupamento semissupervisionado aquele realizado por métodos que incluem mecanismos para a consideração da informação pré-existente no processo de geração de grupos. Os métodos desta categoria de aprendizado podem ser divididos em duas abordagens para incorporação de semissupervisão, dependendo do conhecimento disponível: abordagem baseada em sementes e abordagem baseada em restrições entre pares.

Técnicas baseadas em sementes \cite{Pedrycz1997,Bensaid1996,Bensaid1998,Labzour1998,Basu2002} consideram que uma parte, geralmente pequena, do conjunto de exemplos é rotulada. As sementes, exemplos rotulados do conjunto, podem ser utilizadas de variadas formas, como para estabelecer restrições ao algoritmo, estabelecer restrições entre exemplos/grupos e/ou para definição de rótulos de grupos.

As técnicas baseadas em restrições entre pares \cite{Wagstaff2001,Basu2004,Grira2005,Grira2008} contam com informação prévia na forma de relações entre exemplos que podem ser do tipo \emph{must-link}, indicando que um par de exemplos deve pertencer ao mesmo grupo, ou \emph{cannot-link}, indicando que um par de exemplos deve pertencer a grupos distintos.

O agrupamento \emph{fuzzy} semissupervisionado ocorre quando são incluídos mecanismos de semissupervisão em métodos de agrupamento \emph{fuzzy}. A maior parte das publicações coloca a abordagem de \citeonline{Pedrycz1985} como o primeiro trabalho na área de agrupamento \emph{fuzzy} semissupervisionado. 

\rewrite{A exploração de ------------>>>>>>}

A proposta de métodos semissupervisionados de aprendizado é crescente, uma vez que questões como o volume de dados e o custo de rotulação manual de exemplos persistem.

\citeonline{Hamasuna2011} introduz o conceito de tolerância entre grupos, utilizado em conjunto com restrições entre pares de exemplos para a construção de um novo algoritmo de agrupamento semissupervisionado baseado no FCM. \citeonline{Yan2011} utilizam um conjunto de exemplos rotulados para inicialização e criação de restrições de pares de exemplos, extraídos a partir dos rótulos, durante o processo de agrupamento explorado dentro do contexto de categorização de documentos. O algoritmo \emph{Data Understanding using Semi-Supervised Clustering} \cite{Bhatnagar2012} %não necessita de parâmetros e processa os dados uma única vez, 
utiliza uma porção de exemplos rotulados para a identificação de pequenos grupos dentro das classes. \citeonline{Shamshirband2014} propõem o \emph{D-FICCA}, um algoritmo de agrupamento que integra uma modificação, baseada em densidade e lógica \emph{fuzzy}, para o algoritmo de competição imperialista \cite{Atashpaz2007}. Em \cite{Zhenpeng2014} é proposta uma técnica de agrupamento semissupervisionado baseada no algoritmo $k$-\emph{means} e ganho de informação para escolha dos protótipos incias. O trabalho de \citeonline{Schwenker2014} traz uma revisão atual de outros métodos de agrupamento semissupervisionado.