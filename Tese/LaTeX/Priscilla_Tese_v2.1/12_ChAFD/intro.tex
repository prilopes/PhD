As técnicas clássicas de AM consideram particularidades para os dados disponíveis: assume-se que o conjunto de exemplos é finito e que os exemplos seguem uma distribuição estática e estão disponíveis para acesso sempre que necessário durante o processo de aprendizagem. 

A evolução e ampliação do acesso a novas tecnologias e a internet tornaram propício o surgimento e desenvolvimento de diferentes e novos domínios para os quais as características assumidas pelas abordagens clássicas de AM não são verdadeiras.

Existe hoje uma variedade de sistemas que produzem grande quantidade de dados em curto espaço de tempo, como monitoração de tráfego de rede \cite{Aggarwal2008,Yu2009,Zhang2012,Breve2013}, redes de sensores \cite{gama2007,Pan2007,Zhang2012,Bouchachia2014}, mineração de \emph{clicks} na \emph{web} \cite{Marin2013}, medida de consumo de energia \cite{DeSilva2011,Zhang2012}, fraude de cartão de crédito \cite{wu2012}, mineração de textos da \emph{web} \cite{FdezRiverola2007,Cheng2011,Kmieciak2011,Nahar2014}, rastreamento visual \cite{Liu2014}, olfação artificial \cite{DeVito2012}, pesquisa meteorológica, mercado de ações e registros de supermercados \cite{Yogita2013}.

Sistemas como os citados impulsionaram a pesquisa por técnicas de aprendizado capazes de lidar com as peculiaridades desses novos domínios: tamanho indefinido, potencialmente infinito, e podem gerar exemplos com distribuição estatística mutável de acordo com o tempo \cite{Gama2010}. Nesse contexto teve origem uma nova abordagem denominada Aprendizado em Fluxo de Dados (AFD).

No modelo de Fluxo de Dados (FD) alguns ou todos os exemplos de entrada que serão utilizados não estão disponíveis em disco ou memória para acesso a qualquer momento, mas surgem de maneira contínua, em um ou mais fluxos. FDs diferem de conjuntos de exemplos ditos convencionais em diversos aspectos \cite{Babcock2002}:

\begin{itemize}
\item Os exemplos no fluxo chegam de maneira contínua e constante;
\item O sistema não possui controle sobre a ordem na qual os exemplos chegam para serem processados;
\item Os fluxos têm tamanho potencialmente infinito;
\item Uma vez que um exemplo do FD foi processado, ele é descartado ou arquivado. Estes exemplos não podem ser recuperados de forma simples, pois guardá-los em memória ou disco seria inviável.
\end{itemize}

Devido às limitações de tempo e espaço que ocorrem por causa das peculiaridades de FDs, as técnicas de AFD devem considerar que encontrar conhecimento válido de maneira rápida é uma prioridade para esses domínios, mesmo que o encontrado seja uma aproximação do obtido caso fosse possível ter o conjunto de exemplos completo para a aplicação de AM clássico.