\section{Estruturas de Sumarização de Exemplos} \label{chConceitos:FD:Sumario}

Uma solução para contornar a impossibilidade de armazenamento de todos os exemplos é a criação de sumários ou sinopses da informação encontrada nos dados. Uma grande variedade de técnicas tem sido desenvolvidas para o armazenamento de sumários da informação histórica encontrada em FD. \cite{gama2007}.

É possível manter estatísticas simples de FD, que podem ser computadas de forma incremental. Para definir a média de um FD, por exemplo, precisamos manter o número de observações ($i$) e a soma dos valores encontrados até o momento ($\sum x_i$). Assim, com a chegada de um novo exemplo, a média pode ser calculada de forma incremental, como na \autoref{Eq:mediaRecursiva}.

\begin{equation} \label{Eq:mediaRecursiva}
\bar{x}_{i} = \frac{(i-1) \times \bar{x}_{i-1} +  x_{i}}{i}
\end{equation}

De maneira semelhante podem ser definidas outras estatísticas, como desvio padrão e coeficiente de correlação entre dois fluxos. O interessante nessas fórmulas é poder manter estatísticas exatas sobre uma sequência de dados potencialmente infinita sem ter que armazenar todos os dados \cite{gama2007}.

As estruturas de sumarização mais frequentemente utilizadas por técnicas de agrupamento em FD são detalhadas a seguir.

\paragraph{Vetor de Atributos \\}

O uso de vetores de atributos para sumarização de grandes volumes de dados foi introduzido no algoritmo \emph{BIRCH} \cite{zhang1996}. Este vetor, chamado de \emph{Clustering Feature} (CF), conta com três componentes: o número de exemplos ($N$), a soma linear dos exemplos ($LN$) e a soma quadrática dos exemplos ($SS$), sendo que $LS$ e $SS$ são estruturas $n$-dimensionais, de acordo com o número de atributos do FD. Essas componentes permitem o cálculo de métricas de grupo, como média, raio e diâmetro do grupo. 

O vetor CF possui propriedades de incrementais e aditivas, ou seja, é possível inserir um novo exemplo em um CF pela atualização das estatísticas e dois CF podem ser mesclados em um terceiro vetor CF de forma simples.

Algumas abordagens de aprendizado em FD utilizam o vetor CF como descrito nesta descrição, por vezes incluindo pesos para ponderar os grupos \cite{Cao2006,Kranen2011}. Entretanto, há outras abordagens que utilizam variações do CF, a fim de produzir outras estatísticas.

A estrutura nomeada microgrupo, usada primeiramente no algoritmo \emph{CluStream} \cite{Aggarwal2003}, estende o conceito do vetor CF, adicionando mais duas componentes ao CF original: a soma de marcas temporais ou \emph{timestamps} ($LST$) e a soma quadrática de \emph{timestamps} ($SST$). As duas novas componentes tem o objetivo de incluir aspecto temporal na descrição de grupos, que pode ser utilizado para identificar \emph{outliers} ou desvios de conceito.

A proposta do algoritmo \emph{SWClustering} \cite{Zhou2008} também sugere uma extensão para o vetor CF, chamada de \emph{Temporal CF}, que adiciona uma nova componente ao CF original: a \emph{timestamp} do exemplo mais recente a ser inserido no grupo.

Algoritmos que fazem uso de microgrupos ainda podem manter um histórico dessas estruturas para determinar \emph{snapshots} do FD, i.e., recuperar a situação da partição de grupos em um determinado momento no tempo. \cite{aggarwal2007:Ch2}

\paragraph{Arranjos de Protótipos \\}

Alguns algoritmos de agrupamento utilizam uma estrutura simplificada chamada Arranjos de Protótipos, que consiste em um conjunto de protótipos (medóides, centróides, etc) que sumarizam a partição dos dados.

O algoritmo STREAM \cite{Guha2000} divide o FD em partes (\emph{chunks}) e, para cada uma das partes, são definidos $2k$ exemplos representantes obtidos por uma variante do algoritmo $k$-medóides \cite{Kaufman1990}. Esse processo é repetido até que seja completado um conjunto de $m$ exemplos e, então, o agrupamento é aplicado aos protótipos com o objetivo de reduzir esse conjunto.

Estratégia similar é utilizada para o algoritmo \emph{Stream LSearch} \cite{OCallaghan2002}, que os protótipos em memória. Quando a memória está cheia, o conjunto de protótipos é agrupado a fim de manter na memória apenas um subconjunto de protótipos.

\paragraph{Grades de Dados \\}

A sumarização dos exemplos de um FD também pode ser feita por meio de grades \cite{Cao2006,Chen2007,gama2011}, ou seja, pelo particionamento do espaço $n$-dimensional de atributos em células grade de densidade.

Uma estratégia \cite{Chen2007} para a utilização de grades é a associação de um coeficiente de densidade que decresce com o tempo. A densidade de uma célula de grade é determinada pela soma das densidades de cada exemplo inserido na grade. Cada célula é representada por uma tupla $<tg,tm,D,label,status>$, onde $tg$ é a última vez que a célula foi atualizada, $tm$ é a última vez que a célula foi removida do conjunto de células válidas (não \emph{outliers}), $D$ é a densidade desde a última atualização, $label$ é o rótulo de classe da célula e $status$ indica se é uma célula normal ou esporádica (células com poucos objetos, \emph{outliers}).

A manutenção das células de grade é realizada durante a fase \emph{online}. Uma célula pode se tornar esparsa se não receber exemplos por muito tempo e uma célula esparsa pode se tornar densa se receber muitos exemplos. Quando um novo exemplo chega, é verificado a célula a qual pertence e estrutura da célula é atualizada. Células com o $status$ esporádico são removidas periodicamente.