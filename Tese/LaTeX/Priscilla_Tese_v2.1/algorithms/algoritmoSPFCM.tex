\begin{algorithm2e}[!htb]
	\SetAlgoLined
	\Entrada{$E$, $k$, $m$, $n_{s}$}
	\Saida{$U$, $C$}
	\Inicio{
		$U = $ geraMatrizPertinênciaAleatória()\;
		$C = $ geraCentróidesIniciais($E$, $U$)\;
		$E' = $ gerarSubconjuntosExemplos($E$, $n$)\;
		$w = 1_{n_{s}}$\;
		$U, C = $ WFCM($E'[1]$,$k$,$m$,$w$)\;
		PAREI AQUI \;
		\Enqto{$\epsilon > \xi$}{
			atualizarMatrizPertinência($U$)\;
			$C'= C$\;
			atualizarCentróides($C$)\;
			$\epsilon = max_{1 \leq i \leq k}\{\| c_{i} - c'_{i} \|^{2}\}$\;
		}
	}
	\caption{\emph{Single-Pass Fuzzy} $C$-\emph{Means} (SPFCM) \cite{Hore2007b}} \label{algo:SPFCM}
\end{algorithm2e}

%O vetor $w$ é um conjunto pesos $\{w_{1}, w_{2}, ..., w_{n}\}$, onde $w_{j} \geq 0$ determina a influência de cada exemplo $e_{j}$ para o processo de agrupamento.

%As atualizações para a matriz de pertinência seguem a Equação \ref{eq:fcmAtualizaU} e geração inicial e atualização dos centróides é calculada pela Equação \ref{eq:spfcmAtualizaW} e a

%\begin{equation}
%	c_{i} = \frac{\sum_{j=1}^{n} w_{j}u_{ij}^{m}e_{j}}{\sum_{j=1}^{n} w_{j}u_{ij}^{m}}
%	\label{eq:spfcmAtualizaW}
%\end{equation}