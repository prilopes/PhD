\chapter{Tabela de Depend�ncias}
\label{chap:anexoA}


% OBS: tem que utilizar o pacote \usepackage{multirow}



\begin{table}[htp]
 \begin{footnotesize}\begin{tabular}{|p{7.2cm}|c|p{7.2cm}|}

  
  %\multicolumn{3}{c}{Pr�tica \hfill Depend�ncia \hfill Pr�tica}\\ \hline

  \multicolumn{1}{c}{Pr�tica} & \multicolumn{1}{c}{Dep} & \multicolumn{1}{c}{Pr�tica} \\ \hline




Estabelecer o plano de teste	&	N	&	Monitorar riscos do produto	\\ \hline

Identificar riscos ao projeto de teste	&	N	&	Monitorar riscos do projeto de teste	\\ \hline

Estabelecer uma wbs de alto nivel	&	N	&	Monitorar riscos do projeto de teste	\\ \hline

Definir ciclo de vida de teste	&	N	&	Monitorar riscos do projeto de teste	\\ \hline

Estabelecer o cronograma de teste	&	N	&	Monitorar riscos do projeto de teste	\\ \hline

Identificar caracteristicas nao funcionais a serem testadas	&	N	&	Definir a abordagem de teste nao funcional	\\ \hline

\multirow{2}{*}{Definir categorias e par�metros de risco do produto}	&	N	&	Identificar riscos do produto n�o funcionais	\\ 

	&	N	&	Analisar riscos do produto	\\ \hline

\multirow{2}{*}{Analisar riscos do produto}	&	N	&	Identificar elementos e caracteristicas a serem testados	\\ 

	&	N	&	Definir a abordagem de teste	\\ \hline

Identificar elementos e caracteristicas a serem testados	&	N	&	Definir a abordagem de teste	\\ \hline

\multirow{4}{*}{Definir a abordagem de teste}	&	N	&	Definir criterios de entrada	\\
	&	N	&	Definir criterios de parada	\\ 
	&	N	&	Definir criterios de suspensao e recomeco	\\ 
	&	N	&	Estabelecer uma wbs de alto nivel	\\ \hline
\multirow{2}{*}{Estabelecer uma wbs de alto nivel}	&	N	&	Definir ciclo de vida de teste	\\ 
	&	N	&	Determinar estimativas de esforco e custo de teste	\\ \hline

Definir ciclo de vida de teste	&	N	&	Determinar estimativas de esforco e custo de teste	\\ \hline

Estabelecer uma wbs de alto nivel	&	N	&	Estabelecer o cronograma de teste	\\ \hline

Definir ciclo de vida de teste	&	N	&	Estabelecer o cronograma de teste	\\ \hline

Determinar estimativas de esforco e custo de teste	&	N	&	Estabelecer o cronograma de teste	\\ \hline

Estabelecer o plano de teste	&	N	&	Estabelecer o cronograma de teste	\\ \hline

Estabelecer uma wbs de alto nivel	&	N	&	Planejar a equipe de teste	\\ \hline

\end{tabular} \end{footnotesize}
\end{table}






\begin{table}[!htp]
\begin{footnotesize}\begin{tabular}{|p{7cm}|c|p{7cm}|}
 \multicolumn{3}{c}{Pr�tica \hfill Depend�ncia \hfill Pr�tica}\\ \hline

Determinar estimativas de esforco e custo de teste	&	N	&	Planejar a equipe de teste	\\ \hline

Estabelecer o cronograma de teste	&	N	&	Planejar a equipe de teste	\\ \hline

Estabelecer o plano de teste	&	N	&	Planejar a equipe de teste	\\ \hline

Estabelecer uma wbs de alto nivel	&	N	&	Planejar envolvimento dos interessados	\\ \hline

Definir ciclo de vida de teste	&	N	&	Planejar envolvimento dos interessados	\\ \hline

Definir criterios de entrada	&	N	&	Verificar em relacao aos criterios de entrada	\\ \hline

Estabelecer o plano de teste	&	N	&	Monitorar recursos do ambiente de teste	\\ \hline

Obter comprometimento com o plano de teste	&	N	&	Monitorar compromisssos de teste	\\ \hline

Estabelecer o plano de teste	&	N	&	Conduzir revisoes em marcos do progresso	\\ \hline

Obter comprometimento com o plano de teste	&	N	&	Conduzir revisoes em marcos do progresso	\\ \hline

Identificar riscos ao projeto de teste	&	N	&	Conduzir revisoes em marcos do progresso	\\ \hline

Desenvolver os requisitos do ambiente de teste	&	N	&	Conduzir revisoes em marcos do progresso	\\ \hline

Definir criterios de parada	&	N	&	Monitorar criterios de parada	\\ \hline

Definir a abordagem de teste	&	N	&	Identificar e priorizar condicoes de teste	\\ \hline

\multirow{2}{*}{Analisar riscos do produto}	&	N	&	Identificar e priorizar condicoes de teste	\\ 

	&	N	&	Identificar e priorizar casos de teste	\\ \hline

Definir crit�rios de entrada	&	A	&	Definir criterios de parada	\\ \hline

Definir estrategia de teste	&	A	&	Definir a abordagem de teste	\\ \hline

Estabelecer uma wbs de alto nivel	&	A	&	Definir ciclo de vida de teste	\\ \hline

Definir estrategia de teste	&	A	&	Definir a abordagem de teste nao funcional	\\ \hline

\end{tabular}\end{footnotesize}



\caption{Depend�ncias entre pr�ticas do TMMi identificadas por \citeonline{Hohn}, sendo que N representa depend�ncia ``necess�ria'' e A representa depend�ncia de ``alinhamento''. \label{tab:dependencias}}

\end{table}

