\chapter{Agrupamento em Stream de Dados} \label{agrupamentoDS}

Contexto geral:
\begin{enumerate}
\item Aprendizado - supervisionado, n�o supervisionado, semissupervisionado;
\item Agrupamento dentro de aprendizado
\item Streams de dados - caracter�sticas, dificuldades
\end{enumerate}

\section{Agrupamento de Dados}

Defini��o geral. Agrupamento particional e ref outros tipos.

\begin{enumerate}
\item Processo de agrupamento;
\item defini��es e escolhas: m�tricas, valida��o;
\item k-means;
\item fcm;
\end{enumerate}

\subsection{Agrupamento Semissupervisionado}

Agrupamento com alguma informa��o pr�via.

\begin{enumerate}
\item Tipo de informa��o pr�via;
\item Ref algoritmos.
\end{enumerate}

\section{Aprendizado em Stream de Dados}

\section{T�cnicas de aprendizado de m�quina em Stream de Dados}

Vis�o geral, baseados em t�cnicas comuns atentando aos aspectos espec�ficos do contexto de Streams.

Ref abordagens de "classifica��o".

\subsection{�rvore de Hoedding}

\subsection{Agrupamento em Stream de Dados}
\subsubsection{something else}
\subsubsection{Semissupervisionado}